\subsubsection*{January 23, 2020}

Let $R$ be a ring and $I\subseteq R$ an ideal. Since $I$ is a normal group of $(R, +)$, we can construct the Abelian group
\[\nicefrac{R}{I} = \{r+I\mid r\in R\}.\]
This can be given a ring structure by defining multiplication in the obvious way: 
\[[r][s] = [rs]\]

With $[r] = r+I$, $[s] = s+I$, we get
\begin{align*}
	(r+I)(s+I)&=rs+\underbrace{\smallunderbrace{rI}_{\subseteq I}+\smallunderbrace{sI}_{\subseteq I}+\smallunderbrace{II}_{\subseteq I}}_{=I} \\
	&= rs + I \\
	&= [rs]
\end{align*}

\example If $\varphi: R\to S$ is a homomorphism, $\ker \varphi$ is an ideal in $R$, so this is an extension of $\nicefrac{R}{\ker \varphi}$. (***)

Consider $\nicefrac{\Z_2[x]}{\langle x^2+1 \rangle}$. 

Take the ring of polynomials with coefficients in $\Z_2=\{0,1\}$ and mod out the ideal generated by $x^2 + 1$: 
\[\langle x^2 + 1 \rangle = (x^2+1)\cdot \Z_2[x]\]
In this, two polynomials $p(x)$ and $q(x)$ are equivalent if $p(x)-q(x)\in(x^2+1)\cdot \Z_2[x]$. 

$\implies$ ``$p(x) - q(x)$ is divisible by $x^2+1$'', or ``they have the same remainder under division by $x^2+1$.''

The equivalence classes can be represented by the possible remainders under division by $x^2+1$ $\iff$ all polynomials of degree at most $1$ are in $\Z_2[x]$. 

\[\nicefrac{\Z_2[x]}{\langle x^2+1 \rangle} = \{0, 1, x, x+1\}\]
\begin{center}
\begin{tabular}{c|cccc}
$+$   & $0$   & $1$   & $x$   & $x+1$ \\ \hline
$0$   & $0$   & $1$   & $x$   & $x+1$ \\
$1$   & $1$   & $0$   & $x+1$ & $x$   \\
$x$   & $x$   & $x+1$ & $0$   & $1$   \\
$x+1$ & $x+1$ & $x$   & $1$   & $0$  
\end{tabular}
\hspace{2em}
\begin{tabular}{c|ccc}
$\cdot$ & $1$   & $x$   & $x+1$ \\ \hline
$1$     & $1$   & $x$   & $x+1$ \\
$x$     & $x$   & $1$   & $x+1$ \\œ
$x+1$   & $x+1$ & $x+1$ & $0$  
\end{tabular}
\end{center}

\exercise Give a multiplication table for the ring $\nicefrac{\Z_2[x]}{\langle x^3 + x^2 + x + 1 \rangle}$. 

\begin{center}
\begin{tabular}{c|ccccccc}
$\cdot$       & $1$           & $x$           & $x+1$     & $x^2$         & $x^2 + 1$ & $x^2 + x$ & $x^2 + x + 1$ \\ \hline
$1$           & $1$           & $x$           & $x+1$     & $x^2$         & $x^2 + 1$ & $x^2 + x$ & $x^2 + x + 1$ \\
$x$           & $x$           & $x^2$         & $x^2 + x$ & $x^2 + x + 1$ & $x^2 + 1$ & $x+1$     & $1$           \\
$x+1$         & $x+1$         & $x^2 + x$     & $x^2 + 1$ & $x+1$         & $0$       & $x^2 + 1$ & $x^2 + x$     \\
$x^2$         & $x^2$         & $x^2 + x + 1$ & $x+1$     & $1$           & $x^2+1$   & $x^2+x$   & $x$           \\
$x^2 + 1$     & $x^2 + 1$     & $x^2 + 1$     & $0$       & $x^2+1$       & $0$       & $0$       & $x^2 + 1$     \\
$x^2 + x$     & $x^2 + x$     & $x+1$         & $x^2 + 1$ & $x^2+x$       & $0$       & $x^2 + 1$ & $x + 1$       \\
$x^2 + x + 1$ & $x^2 + x + 1$ & $1$           & $x^2 + x$ & $x$           & $x^2 + 1$ & $x + 1$   & $x^2$        
\end{tabular}
	
\end{center}

\exercise Let $R$ be a ring and $I$ an ideal. If $I$ contains a unit of $R$, $I=R$. 
\begin{proof}
	Let $u\in I$ be a unit. Then it has a multiplicative inverse $u^{-1}\in R$. Then $uu^{-1}\in I\implies 1\in I$. Then $\forall r\in R, 1\cdot r = r\in I$. Thus $I = R$. 
\end{proof}

