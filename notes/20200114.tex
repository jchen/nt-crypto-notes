\subsubsection*{January 14, 2020}

\exercise{13}

\begin{enumerate}[(a)]
	\item Write down the formulas for all homomorphisms from $\Z_6$ into $\Z_9$. 
	
	We can list all the subgroups of $\Z_6$ (recall a kernel has to be a subgroup). Then try to find a homomorphism for each subgroup. 
	
	\begin{itemize}
		\item $\Z_6=\{0,1,2,3,4,5\}:$ $\varphi(m)\equiv 0$
		\item $\{0\}:$ cannot be kernel because there is no subgroup of order $6$ in $\Z_9$
		\item $\{0,2,4\}:$ cannot be kernel
		\item $\{0,3\}:$  $\varphi(m)=3m$
	\end{itemize}

	\item Do the same for all homomorphisms from $\Z_{24}$ into $\Z_{18}$. 
	
	We can attempt to solve this problem in the general sense, classifying all homomorphisms from $\Z_m$ to $\Z_n$. Realize that the generator of the kernel in $\Z_{24}$ must get sent to $0$, i.e. $\equiv 0\mod 18$. This means that any generator must be a common divisor of $24$ and $18$, of which there are $1, 2, 3, $ and $6$. We construct the homomorphism by creating $\varphi(m)=km, k=\frac{18}{g}$ where $g$ is a generator for the kernel. We can extend this to $m$ and $n$. 
\end{enumerate}

\exercise{14} Show that $\phi_3: \Z_3\to \Z_3$ defined by $\phi_3([x])=[x]^3$ and $\phi_5: \Z_5\to \Z_5$ defined by $\phi_5([x])=[x]^5$ are homomorphisms but $\phi_4: \Z_4\to \Z_4$ defined by $\phi_4([x])=[x]^4$ is not. 

\begin{proof}
	We can use the freshman dream lemma: 
	\[(x+y)^p \equiv x^p + y^p\mod p\]
	Which makes any $\phi_p$ a homomorphism. 
\end{proof}

\exercise{15} Let $G$ be an Abelian group, and let $n$ be any positive integer. Show that the function $\phi:G\to G$ defined by $\varphi(x)=nx$ is a homomorphism. 

\begin{proof}
	We write
	\[\varphi(x)+\varphi(y) = nx + ny = \underbrace{x + \cdots + x}_{n\ \text{times}} + \underbrace{y + \cdots + y}_{n\ \text{times}} = \underbrace{(x+y) + \cdots + (x+y)}_{n\ \text{times}}=\varphi(x + y)\]
\end{proof}

\exercise{16} Show that $\varphi:\C^\times \to \R^\times$ defined by $\varphi(a+bi) = a^2+b^2$ is a homomorphism. 

\begin{proof}
We can write it in $e^{i\theta}$ form to make out life easier (otherwise it's algebra and a lot of foiling). 
\[\varphi(re^{i\alpha}se^{i\beta}) = \varphi(rs e^{i(\alpha+\beta)})=r^2s^2\]	
\end{proof}

\exercise{17} Let $\phi$ be a group homomorphism of $G_1$ onto $G_2$. Prove that: 

\begin{enumerate}
	\item If $G_1$ is Abelian then so if $G_2$. 
	
	Let $u,w\in G_2$, $\varphi(x)=u, \varphi(y)=w$ ($\varphi$ is onto)
	\[u+w = \varphi(x) + \varphi(y) = \varphi(x+y) = \varphi(y+x) = \varphi(y) + \varphi(x) =  w + u\]
	Counterexample: $\det : GL_2(\R) \to \R^\times$
	\item If $G_1$ is cyclic then so is $G_2$. 
	
	We take the generator $g_1\in G_1$ and claim it is also a generator $\varphi(g_1) = g_2\in G_2$. 
	
	Counterexample: $\varphi: \Z_2\times \Z_2\to \Z_2$, $\varphi(m,n) = m$. 
	
	Here's another nice counterexample: $\varphi: GL_2(\R)\to \{1\}\subseteq \R^\times$. 
	\item Give a counterexample to the converse of the statement. 
\end{enumerate}

\begin{defn}{Normal Subgroups}
Let $H$ be a subgroup of $G$. $H$ is called \ul{normal} if $ghg^{-1}$ for all $g\in G$, $h\in H$. 
\begin{equation}
gHg^{-1}\subseteq H	
\end{equation}
\end{defn}

\begin{proposition}
	Let $\phi:G_1\to G_2$ be a group homomorphism. 
	\begin{enumerate}[(a)]
		\item If $H_1$ is a subgroup of $G_2$, then $\phi(H_1)$ is a subgroup of $G_2$. If $\phi$ is onto and $H_1$ is normal in $G_1$, then $\phi(H_1)$ is normal in $G_2$. 
		\item If $H_2$ is a subgroup of $G_2$, then $\phi^{-1}(H_2)=\{x\in G_1\mid \phi(x)\in H_2\}$ is a subgroup of $G_1$. If $H_2$ is normal in $G_2$, then $\phi^{-1}(H_2)$ is normal in $G_1$. 
	\end{enumerate}
\end{proposition}

Let's do a quick exercise that involves normal subgroups: 

\exercise{18}
Recall that the center of a group $G$ is $\{x\in G\mid xg=gx\text{ for all }g\in G\}$. Prove that the center of any group is a normal subgroup. 

\begin{proof}
	Take $x\in Z(G)$ and $g\in G$. 
	\[gxg^{-1} = xgg^{-1} = x\in Z(G)\]
\end{proof}
