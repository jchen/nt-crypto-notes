\subsubsection*{January 10, 2020}
\subsection{Group Homomorphisms}
\begin{defn}{Group Homomorphism}
Let $G_1, G_2$ be groups. Then $\varphi: G_1\to G_2$ is called a \ul{group homomorphism} if
\begin{equation}
	\varphi(xy) = \varphi(x) \varphi(y)
\end{equation}
for all $x,y\in G_1$. 
\end{defn}
\begin{defn}{Additional Terminology}
There are some additional classifications of homomorphisms: 

$\varphi$ is called an \ul{isomorphism} if $\varphi$ is 1-1 and onto. 

$\varphi$ is called \ul{automorphism} if $\varphi$ is an isomorphism and $G_1=G_2$
\end{defn}

\begin{defn}{Kernel and Image}
Similar to linear algebra and linear transformations, we have \ul{kernel} and \ul{image}. 
\begin{align*}
\ker \varphi &\Def \{x\in G_1\mid \varphi(x)=e_2\} \text{ where $e_2$ is the identity in $G_2$} \\
\mathrm{im} \varphi &\Def \{\varphi(x)\mid x\in G_1\}
\end{align*}
We call $G_2$ the ``codomain'' of $\varphi$. 
\end{defn}

\begin{proposition}
	Let $\varphi: G_1\to G_2$ be a homomorphism. $e_1$ the identity of $G_1$ and $e_2$ the identity of $G_2$. Then
	\begin{enumerate}[(i)]
		\item $\varphi(e_1)=e_2$,
		\item $\varphi(x^{-1}) = \varphi(x)^{-1}$ for all $x\in G_1$.
	\end{enumerate}
\end{proposition}

\begin{proof}
We use the homomorphism definition. 
\begin{enumerate}[(i)]
	\item $\varphi(e_1)\cdot \varphi(e_1) = \varphi (e_1^2) = \varphi (e_1) = \varphi (e_1)\cdot e_2\implies \varphi(e_1)=e_2$
	\item $\varphi(x^{-1})\cdot \varphi(x) = \varphi(x^{-1}\cdot x) = \varphi(e_1) = e_2\implies \varphi(x^{-1}) = \varphi(x)^{-1}$
\end{enumerate}
\end{proof}

\begin{proposition}
\!
\begin{enumerate}[(i)]
	\item $\ker \varphi$ is a subgroup of $G_1$, 
	\item $\mathrm{im} \varphi$ is a subgroup of $G_2$. 
\end{enumerate}	
\end{proposition}

\begin{proof}
\!
\begin{enumerate}[(i)]
	\item Let $x, y\in \ker \varphi$. Then $\varphi(xy)=\varphi(x)\varphi(y)=e_2\cdot e_2=e_2\implies xy\in \ker \varphi$. 
	\item Let $u, w\in \mathrm{im} \varphi$ such that $\varphi(x) = u, \varphi(y) = w$. This implies $uw = \varphi(x)\varphi(y) = \varphi(xy)\implies uw\in \mathrm{im}\varphi$. 
\end{enumerate}	
\end{proof}

We also state the following without proof: 
\begin{itemize}
\item The inverse of an isomorphism is an isomorphism. 
\item The composition of isomorphisms is an isomorphism. 
\item We say $G_1\cong G_2$ if there exists an isomorphism $\varphi: G_1\leftrightarrow G_2$, and isomorphisms of groups in an equivalence relation. 	
\end{itemize}

\subsection{Exercises}

\exercise{10} Let $G$ be the following set of matrices over $\R$: 
\[\mtrx{1 & 0 \\ 0 & 1}, \quad \mtrx{1 & 0 \\ 0 & -1}, \quad \mtrx{-1 & 0 \\ 0 & 1}, \quad \mtrx{-1 & 0 \\ 0 & -1}\]
Show that $G$ is isomorphic to $\Z_2\times \Z_2$. 
\begin{proof}
We can construct the map 
\[\mtrx{1 & 0 \\ 0 & 1} \mapsto (0,0), \quad \mtrx{1 & 0 \\ 0 & -1}  \mapsto (0, 1), \quad \mtrx{-1 & 0 \\ 0 & 1}  \mapsto (1, 0), \quad \mtrx{-1 & 0 \\ 0 & -1}  \mapsto (1,1)\]
Constructing a Cayley table (or simple trial and error) shows that this is a valid isomorphism. Another things we could note is that the only groups with $4$ elements are isomorphisms of $\Z_4$ or $\Z_2\times\Z_2$. Noting that the matrices aren't cyclic (there is no generator) also gives the compatible conclusion. 
\end{proof}

\exercise{11}
Let $G$ be any group. and let $a$ be a fixed element of $G$. Define a function $\phi_a: G\to G$ by $\phi_a(x)=axa^{-1}$, for all $x\in G$. Show that $\phi_a$ is an isomorphism. 
\begin{proof}
We first show that $\phi_a$ is a bijection by proving that it's 1-1 and onto:
\begin{align*}
	\phi_a(x) &= \phi_a(y) \\
	axa^{-1} &= aya^{-1} \\
	x &= y
\end{align*}
Let $w\in G$. Then $\phi_a(a^{-1}wa) = aa^{-1}waa^{-1} = w$. 

We now show that $\phi_a$ is a homomorphism: 
\[\phi_a(x)\phi_a(y) = axa^{-1}aya^{-1} = axya^{-1} = \phi_a(xy)\]
\end{proof}

\begin{proposition}
Let $\varphi\in \mathrm{Hom}(G_1, G_2)$. Then $\varphi$ is 1-1 iff $\ker \varphi = \{e_1\}$.
\end{proposition}
\begin{proof}
($\Rightarrow$) If $\varphi$ is 1-1, and $x\in \ker \varphi$, 
\[\varphi(x) = \varphi(e_1)=e_2 \implies x=e_1\]	

($\Leftarrow$) If $\ker \varphi = \{e_1\}$ and
\begin{align*}
\varphi(x) &= \varphi(y) \\
\varphi(x)\varphi(y)^{-1} &= e_2 \\
	\varphi(x)\varphi(y^{-1}) &= e_2 \\
	\varphi(xy^{-1}) &= e_2 \\
	xy^{-1} &= e_1 \\
	x &= y
\end{align*}
\end{proof}

\exercise{12} Show that the multiplicative group $\Z_7^*$ is isomorphic to the additive group $\Z_6$. 
\begin{proof}
By trial and error, we find that $3$ is a generator in $\Z_7^*$. 
\[3^1 = 3, \quad 3^2 = 2, \quad 3^3 = 6, \quad 3^4 = 4, \quad 3^5 = 5, \quad 3^6 = 1\]
So we map powers of $3$ to its powers which is an isomorphism. 
\end{proof}


