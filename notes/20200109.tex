\subsubsection*{January 9, 2020}
\subsection*{Quick Summary}
We were talking about \ul{groups}, which are
\begin{itemize}
	\item associative,
	\item have an identity element,
	\item and have inverses.
\end{itemize}

There can be \ul{subgroups} in groups, a subset of a group that is also a group. 

An \ul{Abelian group} is a commutative group. 

A \ul{cyclic group} is generated by an element $g$, they're always Abelian. 

\ul{Lagrange's Theorem}: the order of a subgroup divides the order of the group. 

\ul{Index} of $H$ in $G$: $[G:H]=\frac{|G|}{|H|}$. 

\example
\begin{itemize}
\item $(\R^n, +), (\C^n, +)$. 
\item $(\Z_n, +), (\Z_n^*, \cdot), (\R, +), (\R^*, \cdot), (\C, +), (\C^*, \cdot)$. 
\item $GL_n(\R)$: Invertible $n\times n$ matrices. 
\item $SL_n(\R)$: Subgroup of $GL_n(\R)$ with determinant 1. 
\item $GL_n(\C)$: Invertible $n\times n$ complex matrices. 
\item $U(n)$: Unitary group – determinant of absolute value 1. 
\item Symmetry groups of geometric shapes. (Dihedral groups)
\item Frieze groups. 
\item Wallpaper groups. 
\item Crystallographic groups. 
\item Permutation groups of $\{1, \dots, n\}$ under composition ($S_n$), and its subgroups. 
\end{itemize}

We will be focusing mainly on $(\Z_n, +), (\Z_n^*, \cdot)$. 

\begin{defn}{Order of an element}
Let $G$ be a group and $g\in G$. Then the \ul{order} of $o(g)$ is the smallest positive integer $n$ such that $g^n=e$. (May be infinite)
\end{defn}

\begin{proposition}
	Every group of prime order is cyclic. 
\end{proposition}
\begin{proof}
Let $e\neq g\in G$. Consider the subgroup $\langle g \rangle =\{g^n\mid n\in \Z\}$ generated by $g$. Then $|\langle g \rangle|$ divides $|G|$, which is prime. Hence $\langle g \rangle = G$. 
\end{proof}

\subsection{Exercises}

\exercise{1}
Show that $\{(1), (1, 2)(3, 4), (1, 3)(2, 4), (1, 4)(2, 3)\}$ is a subgroup of $S_4$. 

\begin{proof}
We can construct a Cayley table for the subgroup. (Top row is evaluated first). 

\begin{center}
\begin{tabular}{c|cccc}
$\circ$        & $(1)$          & $(1, 2)(3, 4)$ & $(1, 3)(2, 4)$ & $(1, 4)(2, 3)$ \\ \hline
$(1)$          & $(1)$          & $(1, 2)(3, 4)$ & $(1, 3)(2, 4)$ & $(1, 4)(2, 3)$ \\
$(1, 2)(3, 4)$ & $(1, 2)(3, 4)$ & $(1)$          & $(1, 4)(2, 3)$ & $(1, 3)(2, 4)$ \\
$(1, 3)(2, 4)$ & $(1, 3)(2, 4)$ & $(1, 4)(2, 3)$ & $(1)$          & $(1, 2)(3, 4)$ \\
$(1, 4)(2, 3)$ & $(1, 4)(2, 3)$ & $(1, 3)(2, 4)$ & $(1, 2)(3, 4)$ & $(1)$         
\end{tabular}
\end{center}

Every element is its own inverse, and $(1)$ is the identity element. It is evident from the table that we also have closure. Hence, the set is a valid subgroup under composition. 
\end{proof}

\exercise{2}
Let $G$ be an Abelian group. Show that the set of all elements of $G$ of finite order forms a subgroup of $G$. 

\begin{proof}
Let $H$ be the subset of $G$ with elements of finite order. We want to show that $\forall x, y\in H$, $xy^{-1}\in H$. In other words, $xy^{-1}$ also has finite order. Well, $xy^{-1}$ has order at most $k=\mathrm{lcm}(o(x), o(y))$ such that 
\begin{align*}
	(xy^{-1})^k &= x^k (y^{-1})^k \\
	&= x^k (y^k)^{-1} \\
	&= e\cdot e^{-1} \\
	&= e
\end{align*}
Thus, $H$ is a subgroup of $G$. 
\end{proof}

\exercise{3}
Let $G$ be a group. Define the set $Z(G)=\{x\in G\mid xg=gx\text{ for all }g\in G\}$ of all elements that commute with every other element of $G$ is called the \ul{center} of $G$. 
\begin{enumerate}[(a)]
\item Show that $Z(G)$ is a subgroup of $G$. 
\item Show that $Z(G) = \cap_{a\in G}C(a)$. 
\item Compute the center of $S_3$. 
\end{enumerate}

\begin{proof}
\begin{enumerate}[(a)]
\item $e$ by definition commutes with every other element. $\forall x, y\in Z(G)$, $xy\in Z(G)$ as $x,y$ commutes with every element. 
\[(xy)a = x(ya) = x(ay) = (ay)x = a(yx) = a(xy)\]

Inverses also exist as
\[x^{-1}a = (a^{-1}x)^{-1} = (xa^{-1})^{-1} = ax^{-1}\]

\item If an element is in the intersection of all those sets, then it commutes with every element. 
\item Realize this is simply $D_3$, where only the identity commutes with one another. \answer{$\{e\}$}
\end{enumerate}
	
\end{proof}


