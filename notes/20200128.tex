\subsubsection*{January 28, 2020}

Last time: If $R$ is a ring without proper ideals, then it is a field. 

\begin{proposition}
	Let $R$ be a ring and $I$ an ideal. Then there is a 1-1 correspondence between the ideals in $\nicefrac{R}{I}$ and the ideals in $R$ containing $I$. 
\end{proposition}
\begin{corollary}
	If $I$ is a maximal ideal in $R$, then $\nicefrac{R}{I}$ is a field. 
\end{corollary}
\begin{proof}
	Let $I$ be an ideal in $R$, and $J$ an ideal with $I\subseteq J \subseteq R$. Consider the set $\{[j]\mid j\in J\}\subseteq \nicefrac{R}{I}$. We will show that this is an ideal in $\nicefrac{R}{I}$. 
	\begin{align*}
		\text{Let }[j_1],[j_2]&\in K, [r]\in \nicefrac{R}{I} \\
		[j_1]+[j_2]&=\underbrace{[j_1+j_2]}_{\in J}\in K \\
		[j_1][r]&=\underbrace{[j_1r]}_{\in J}\in K
	\end{align*}
	Conversely, let $K$ be an ideal in $\nicefrac{R}{I}$. We show that this corresponds to an ideal $J$ in $R$ with $I\subseteq J$: 
	Let $J=\{a\in R\mid [a]\in K\}$
	
	$\text{Let }a,b \in J\text{. Then }[a],[b]\in k$
	
	$\quad \text{so }[a] + [b] = [a + b]\in K \implies a + b \in J$
	
	$\text{Let }a\in J, r\in R\text{. Then}$
	
	$\quad [a][r]\in K$ ($K$ is an ideal)
	
	$\implies [ar]\in K$
	
	$\implies ar\in J$
	\end{proof}

Finally, we characterise the maximal ideals in the rings we are interested in: $\Z$, $\F[x]$ (the set of polynomials with coefficients in field $\F$). 

These rings are principal ideal domains: all their ideals are generated by one element. If $I$ is an ideal in a principal ideal domain $R$, then $I=aR$ for some $a\in R$. In a p.i.d., the maximal ideal are precisely the prime ideals---so each maximal ideal is generated by an irreducible element of $R$. 

\begin{theorem}
	Every nonzero prime ideal of a principal ideal domain is maximal. 
\end{theorem}

\example $\Z$ is a principal ideal domain. 
\begin{itemize}
	\item $Z$ is an integral domain because $m\cdot n = 0 \implies m = 0$ or $n=0$. 
	\item Let $I=a\Z + b\Z$. Then $I$ is an ideal in $\Z$, and we show that $\exists n\in Z$ such that $I=n\Z$. Then $\boxed{n=g.c.d(a,b)}$. 
\end{itemize}

\exercise Let $P$ be a prime ideal of the commutative ring $R$. Prove that if $I$ and $J$ are ideals of $R$ and $I\cap J\subseteq P$, then either $I\in P$ or $J\in P$. 
\begin{proof}
	Let $I,J$ be ideals, $P$ prime ideal. $I\cap J\subseteq P$. 
	
	Suppose $I\not\subseteq P$. We show that $J\subseteq P$. 
	
	Let $j\in J$, and $i\in I$, $i\not\in P$. THen $ji\in I\cap J\subseteq P$. 
	
	$\quad \implies ji\in P \implies j \in P$ or $i\in P$ (by definition of prime ideal). 
	
	Since $i\not\in p$, $j\in P$. 
\end{proof}

\exercise Find a nonzero prime ideal of $\Z \oplus \Z$ that is not maximal. 
\begin{proof}
	Recall that $\Z\oplus \Z = \{(m,n)\mid m,n\in \Z\}$. 
	\[(m_1,n_1)+(m_2,n_2)=(m_1+m_2,n_1+n_2)\]
	\[(m_1,n_1)\cdot(m_2,n_2)=(m_1\cdot m_2,n_1\cdot n_2)\]
	$P$ is a prime ideal if for $rs\in P$, $r\in P$ or $s\in P$. 
	$I=p\Z \oplus p\Z$, $p$ prime. 
	
	$I$ is a prime ideal but not maximal because $I\subsetneqq \Z\oplus p\Z\subsetneqq \Z \oplus \Z$. 
\end{proof}

\exercise Let $R$ be a commutative ring, with $a\in R$. The annihilator of $a$ is defined by 
\[\mathrm{Ann}(a)=\{x\in R\mid xa=0\}\]
Prove that $\mathrm{Ann}(a)$ is an ideal of $R$. 
\begin{proof}
	Let $x,y\in \mathrm{Ann}(a)$. 
	\[(x+y)a=xa+ya = 0\implies x+y\in R\]
	Let $r\in R$. 
	\[(xr)a = r(xa) = r\cdot 0 = 0\implies xr \in R\]
\end{proof}

\exercise Let $I$ be the smallest ideal of $\Z[x]$ that contains both $2$ and $x$. Show that $I$ is not a principal ideal. 
\begin{proof}
	If $I$ were a principal ideal, then $\exists a,m,n\in \Z[x]$ such that $am=2$ and $an=x$. $2$ and $x$ are irreducible, so either $a = 2$ or $m=2$ (and likewise). Realize this forces $m=2$, $n=x$, and $a=1$ which is not the smallest ideal of $\Z[x]$ that contains both $2$ and $x$. 
\end{proof}