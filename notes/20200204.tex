\subsubsection*{February 4, 2020}
\section{Primality Testing}
\begin{theorem}\textbf{(Fermat's Little Theorem).}
	\begin{equation}
		a^{p-1}\equiv 1\pmod{p}
	\end{equation}
\end{theorem}

\ul{Note:} We let $\F_p$ denote the field that is $\Z_p$ under addition and $\Z_p^*$ under multiplication. 

In $\F_p$, F$\ell$T can be written as
\begin{equation}
	[a]^{p-1}=1
\end{equation}
and it can be proved easily using group theory: 

\begin{proof}
Consider $\langle a\rangle$ in $\F_p$. 
\begin{align*}
	\text{Then }\mathrm{ord}(a) &= \text{smallest integer $n$ such that $a^n=1$} \\
	&= \text{order of $\langle a\rangle$}
\end{align*}
Since there are $p-1$ elements in $\Z_p^*$, $\mathrm{ord}(a)\mid p-1$. Hence $\exists k$ such that $\mathrm{ord}(a)\cdot k=p-1$. Therefore $a^{p-1}=a^{\mathrm{ord}(a)\cdot k}=\left(a^{\mathrm{ord}(a)} \right)^k =1^k = 1$. 
\end{proof}

\begin{defn}{Witness}
	If $p$ is not a prime, it is not necessarily true that $a^{p-1}\equiv 1\pmod(p)$. In that case we call ``$a$'' a \ul{witness} for the compositeness of $p$. 
\end{defn}

This could be used for primality checking if it weren't for the fact that some composite numbers don't have any witnesses: 
\[a^{p-1}\equiv 1\pmod{p}\]
even though $p$ is not a prime. These are called ``Carmichael numbers''. 

Miller-Rabin is a test for compositeness in which each composite number has a lot of witnesses. While not waterproof, it can be used to check for primality by applying it to a large number of potential witnesses. Each time the test fails, it strengthens the evidence for $p$ being prime. 

\begin{proposition}
	Let $p$ be an odd prime and write
	\[p-1=2^kq\qquad\text{\textit{with $q$ odd.}}\]
	Let $a$ be any number not disivible by $p$. Then one of the following two assertions is true: 
	\begin{enumerate}[(i)]
		\item $a^q$ is congruent to $1$ modulo $p$. 
		\item One of $a^q, a^{2q}, a^{4q}, \dots, a^{2^{k-1}q}$ is congruent to $-1$ modulo $p$. 
	\end{enumerate}
\end{proposition}

\begin{proposition}
	Let $n$ be an odd composite number. Then at least $75\%$ of the numbers $a$ between $1$ and $n-1$ are Miller-Rabin witnesses for $n$. 
\end{proposition}
